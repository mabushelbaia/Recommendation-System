\chapter{Requirements and Literature Review}
\minitoc 

\section{Functional Requirements}

The system should provide a RESTful API as the final interface to be used by the front-end application.
The API provides endpoints that allow inserting customers, products, and interactions. In addition to endpoints for retrieving the recommendations for a given customer.

\section{System Requirements}

In order for the system to be useful it has to meet the following specifications:

\subsection{Scalability}
Scalability implies that it has to be cloud-native, the inference system should apply proper load balancing across multi-node, multi-model deployments.

\subsection{Near Real-time Predictions}
To be usable in any website or application, the system should be able to provide real-time predictions, and suggestions, with a few hundred milliseconds latency. \\

To fulfill this requirement, trained models should run on optimized inference servers or services, the suggested deployment plan is to use
\textbf{Nvidia Triton}
\footnote{Nvidia Triton Inference Server, part of the Nvidia AI platform and available with Nvidia AI Enterprise, is open-source software that standardizes AI model deployment and execution across every workload~\cite{Triton}}. 
inference server~\cite{Triton}, 
integrated with \textbf{Amazon SageMaker} model deployment~\cite{SageMaker} as infrastructure.

\subsection{Continuous Training and Deployment}

This implies continuous training and deployment of the model which requires the automation of training steps and deployment.

\subsection{Elasticity and Optimization}

Elasticity is vital for keeping up with traffic spikes and declines while optimizing infrastructure costs. To achieve this, the system should be able to scale up and down based on the traffic and load.

\subsection{Security}

Like any other system, the system has to be immune to security threats by implementing best practices at every level in the deployment and design. \\ \\
For example, rate-limiting requests to interaction injection endpoints, using attestation when possible, and limiting access to user and product CRUD operations.
 