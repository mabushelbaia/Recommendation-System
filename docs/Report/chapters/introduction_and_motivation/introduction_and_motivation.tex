\chapter{Introduction}
\minitoc

\section{Motivation}

The exponential growth of e-commerce has introduced an enormous amount of choice, where consumers face overwhelming product options. To address this challenge, personalized recommendation systems~\cite{raghavendra2018personalized} have become essential for enhancing the shopping experience and increasing the conversion rate for any e-commerce platform.

In contrast to conventional collaborative filtering~\cite{NvidiaRecSys}, content-based~\cite{pazzani2007content}, or popularity-based recommendation systems, our AI-based solution offers distinct advantages. Firstly, AI makes it possible to provide per-user personalized recommendations, which are tailored to their unique preferences and behaviors, enhancing user engagement and satisfaction. AI systems can also intelligently recommend comparable or complementary products or content to increase revenue through cross-selling. Furthermore, AI takes into account the impressions and interactions of users with items, allowing for a more dynamic and accurate understanding of user preferences. Using AI leads to improved recommendation accuracy and relevancy, leading to increased conversion rates and business growth.
    




Statistics from different use cases of recommendation systems:
\begin{itemize}[left=0in]
    \item An intelligent recommender system delivers on average a
    \underline{22.66\% lift in conversions rates}~\cite{salesforce2014predictive} for web products.
    
    \item IKEA experienced a \underline{30\% increase in click-through rate, 2\% surge in average order value}~\cite{IkeaRecAtGoogleCloudSummit} using Google Recommendations AI~\cite{GoogleRecommendationsAI}.
    
    \item Lotte Mart experienced a
    \underline{1.7x increase in new product purchases} ~\cite{LotteMartAwsPersonalize} using Amazon Personalize~\cite{AWSPersonalize}.
\end{itemize}
In summary, the project's motivation is elevating the e-commerce experience, driving business success, and harnessing cutting-edge AI technologies to create a recommendation system that is both high-performing and scalable.

\section{Problem Statement}

The process of building the solution is mainly two parts:

\begin{itemize}
    \item First, designing a personalized recommendation system that covers what traditional collaborative filtering, content-based, or popularity-based systems cannot achieve.
    \item Second, deploying and automating the solution, including, data cleaning, data storage, and model deployment processes, and ensuring a production-ready and scalable system.
\end{itemize}
\section{Report Organization}

The rest of the report is organized as follows. 

Chapter 2 delves into recommendation systems and their components, and reviews related work. 
Chapter 3 outlines functional and system requirements alongside 
a literature review of the existing
recommendation systems and libraries. Chapter 4 unveils the proposed solution, 
its components, and the recommendation pipeline with its stages. 
It also discusses its deployment and infrastructure. 
Chapter 5 presents an experiment that evaluates the proposed solution and its results, and analyzes their implications. 
Finally, Chapter 6 
concludes with key findings and outlines promising avenues for 
future exploration.