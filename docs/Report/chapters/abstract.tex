\cleardoublepage \phantomsection \addcontentsline{toc}{chapter}{English Abstract} \mtcaddchapter
\chapter*{Abstract}
The project aims to design and develop a cutting-edge accelerated e-commerce deep learning recommendation system. The goal is to deliver a production-ready solution, with automated data injection and training pipelines, and a simple RESTful application programming interface (API) as a final interface. The project will have a special focus on scalability and performance.

This report discusses different types of recommendation systems and then compares them to deep learning recommendation model (DLRM) based systems in terms of different metrics and features, such as their accuracy, scalability, and performance.
Furthermore, it compares existing solutions and their aspects, in addition to discussing possible technolgies and architectures to use in the system.
\cleardoublepage \phantomsection \addcontentsline{toc}{chapter}{Arabic Abstract} \mtcaddchapter
\chapter*{\flushright{\RL{المستخلص}}}
\begin{RLtext}
يهدف المشروع إلى تصميم وتطوير نظام توصية لمنصات التجارة الإلكترونية باستعمال التعلم الآلي العميق. الهدف النهائي هو تقديم حلول صالحة لبيئة التشغيل، تتم فيها أتمتة عمليات إدخال البيانات و تدريب نماذج التعلم اللآلي وواجهة برمجة تطبيقات 
\LR{RESTful API}
 كواجهة نهائية. سيركز المشروع بشكل خاص على قابلية التوسع والأداء.

يناقش هذا التقرير أنواعًا مختلفة من أنظمة التوصيات ويقارنها بالأنظمة القائمة على نماذج التوصية بالتعلم اللآلي العميق (
    \LR{DLRM}
    ) من حيث المقاييس والميزات المختلفة. علاوة على ذلك، فهو يقارن الحلول المتوفرة حالياً و مزاياها، كما ويناقش التقنيات والبنى الممكن استعمالها في تطوير النظام.
\end{RLtext}

\justifying