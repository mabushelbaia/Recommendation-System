\chapter{Conclusion and Future Work}
\minitoc

\section{Conclusion}
In conclusion, the development and deployment of a cutting-edge accelerated e-commerce deep learning recommendation system present both challenges and opportunities. Through the exploration of various recommendation system types, it is evident that DLRMs offer promising potential in terms of scalability and performance. However, the evaluation of the AliCCP dataset suggests that while these models perform similarly in terms of AUC metrics, they still face the challenge of effectively recommending relevant items from vast databases.

Despite the seemingly modest AUC scores around 0.5, it's crucial to contextualize these results within the practical implications of recommendation systems. A 50\% chance of user interaction with recommended items demonstrates considerable success, especially considering the volume of choices users encounter. Even a single click on a recommended item signifies the system's effectiveness in guiding user engagement.

Moving forward, the focus lies on the efficient deployment and automation of the recommendation pipeline, including data injection, training, and serving recommendations through a simple RESTful API interface. Optimization of ranking and retrieval models remains paramount to enhance the system's accuracy and user satisfaction.

\section{Future Work}
Build and implement the system's components, deploy and automate the recommendation pipeline and required infrastructure 
In addition to optimizing the ranking and retrieval models
There are many possible improvements and future work that can be done to the system, 
some of them are related to the model and data, and others are related to the system architecture and infrastructure.

The following sections will discuss some of the possible future work and improvements that can be made to improve the system.

\subsection{Deeper Product Features}

One of the main limitations of the current system is the lack of product features related to the product's content, such as the product's description, title, and images.
Adding these features to the model can improve the recommendation system's accuracy and performance. 
But to do that, the system needs to be able to process and analyze the product's content, which requires additional layers of data processing and feature extraction.

As an example, to extract features from the product's name and description, the system can use NLP
techniques to extract embeddings representing the product's content, and then use these embeddings as input features to the model.

For images, the system can use computer vision techniques, such as convolutional networks, to extract features from the product's images, and then use these features as input to the model.

\subsection{Increasing Interactions Data}

The current system recommends products solely based on a single type of interaction, which is the user's click history.
Adding more types of interactions, such as the user's purchase history, the user's rating history, and the user's browsing history, 
might improve the recommendation system's accuracy and performance.

Moreover, some interaction types might be converted to numeric features for further analysis,
 for example, a user opens the product page, and stays for a certain amount of time, 
 that time can be used as a feature to represent the user's interest in the product.

 \subsection{Integrating with Ecommerce Platforms}

The current system is a standalone system that provides an API for the client to interact with.
To make the system more accessible and easier to use, it can be integrated with popular e-commerce platforms, such as Shopify, Magento, and WooCommerce.
Such integration can be done by providing plugins or extensions that can be installed on the e-commerce platform, 
and then the plugin can communicate with the system's API to get recommendations for the platform's users.

\subsection{Session-based Recommendations}

The current system requires re-training to update the model with new data, and it does not take into account the user's current session or context.
So the customer's interactions do not affect the recommendations until the next re-training cycle.

A major improvement to the system is to add a session-based recommendation model that can provide real-time recommendations based on the user's current session and context.
Such a model can utilize RNN with LSTM to capture the user's session and context without requiring prior knowledge of the interactions during training. 

One possible library to use for this purpose is Merlin Transformer4Rec \cite{NvidiaMerlinTransformers4Rec}, 
which is a flexible and efficient library for sequential and session-based recommendation \cite{NvidiaMerlinTransformers4Rec}.

